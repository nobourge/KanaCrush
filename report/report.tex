\documentclass{article}
\usepackage[utf8]{inputenc}
\usepackage[T1]{fontenc}
\usepackage[french]{babel}
\usepackage[parfill]{parskip}
\usepackage{amsmath}
\usepackage{amssymb}
\usepackage{amsfonts}
\usepackage{graphicx}
\usepackage{subfigure}
\usepackage[font={small}]{caption}
\usepackage{float}
\usepackage{listingsutf8}
\usepackage{fullpage}
\usepackage[nochapter]{vhistory}
\usepackage{hyperref}
\usepackage{titlesec}
\usepackage{xcolor}
\usepackage{verbatim}
\usepackage{graphicx}
\usepackage{subcaption}
\usepackage{comment}

\usepackage{natbib}
\usepackage{url}
\usepackage{algpseudocode}

\usepackage{adjustbox}




\newcommand*{\MyIncludeGraphicsMaxSize}[2][]{%
\begin{adjustbox}{max size={\textwidth}{\textheight}}
    \includegraphics[#1]{#2}%
\end{adjustbox}
}
\usepackage{array,booktabs,ragged2e}
\newcolumntype{R}[1]{>{\RaggedLeft\arraybackslash}p{#1}}
\newcolumntype{D}[1]{>{\RaggedLeft\arraybackslash}p{#1}}

% -----------------------------------------------------
% -----------------------------------------------------
% -----------------------------------------------------

\hypersetup{
%couleurs des liens cliquable changée pour une meilleur lisibilité
    colorlinks=true,
    linkcolor=blue,
    filecolor=magenta,
    urlcolor=cyan,
    pdfpagemode=FullScreen,
    }


\title{kanacrush}
\author{Neon }
\date{August 2022}

\begin{document}

\maketitle
\tableofcontents
\newpage


\section{Introduction}
\section{Tâches }
 Indiquez les tâches que vous avez accomplies.

\subsection{1. Fonctionnalité de base}
\subsection{4. Faire glisser le carré}
\subsection{14. Éditeur de tableau.  }
Ajoutez la possibilité de pouvoir modifier un niveau de manière
interactive au lieu d’y jouer. Ce n’est évidemment pas dans le jeu original ! Vous devriez
pouvoir enregistrer les modifications et ajouter de nouveaux niveaux.
\section{Classes }
Pour chaque classe, voici l’interface (pas le corps des méthodes) et  le rôle de chaque classe et comment elle se rapporte aux autres classes.
\section{Logique du jeu}
Supposons que le jeu démarre,  le premier niveau est sélectionné sélectionné (si la
tâche 11 est terminée), attend  10 secondes, puis fais un coup. Décrivez en détail ce qui
se passe dans votre code.
\section{Modèle-Vue-Contrôleur}
Avez-vous utilisé ce modèle de conception ? Si c’est le cas, expliquez comment vos classes correspondent à ce modèle de conception.

\section{Score}
ous calculezon le score.


\section{ressources}




L'écriture du code a été accélérée à l'aide du plugin "Github Copilot"
\href{https://copilot.github.com/}{https://copilot.github.com/}


\end{document}
